\chapter{Zusammenfassung und Ausblick}\label{Kap5}
\begin{quote}
\onehalfspacing
I have converted my Classic Benchmarks to run on the Linux based Raspberry Pi. These are Whetstone, [...], Linpack and Livermore Loops. [...] The Livermore Loops benchmark was used to accept the first supercomputer. So the main bragging rights are:

In 1978, the Cray 1 supercomputer cost \$7 Million, weighed 10,500 pounds and had a 115 kilowatt power supply. It was, by far, the fastest computer in the world. The Raspberry Pi costs around \$70 (CPU board, case, power supply, SD card), weighs a few ounces, uses a 5 watt power supply and is more than 4.5 times faster than the Cray 1. 

My bragging rights are that I developed and ran benchmarks, including Whetstones, on Serial 1 Cray 1 (vgl. \url{http://www.raspberrypi.org/forum/viewtopic.php?f=31&t=44080}).
\end{quote}
In der vorliegenden Arbeit wurde das Skalierungsverhalten eines Raspberry Pi-Beowulf-Clusters mit 20 RPi-Knoten unter Verwendung ausgew"ahlter HPC-Benchmarks untersucht. Als Parameter wurden Performance (Ausf"uhrungsrate und Ausf"uhrungsdauer) und Stromverbrauch betrachtet. 

Der RPi-Cluster mit der beschriebenen Architektur war von der Lehr-und Forschungseinheit f"ur Kommunikationssysteme und Systemprogrammierung zur Verf"ugung gestellt worden. Als Arbeitslast-Generatoren waren HPL und STREAM vorgegeben worden. Zur Integration der durchgef"uhrten ExperimentSuites in einen gr"o\ss eren Versuchsaufbau war ein Datenbankschema vorgegeben worden. Es wurde w"ahrend der praktischen Arbeit aktualisiert. Ebenso wurden Fehlerf"alle bez"uglich Hardware und Systemkonfiguration laufend untersucht und weitestgehend eliminiert. 

Die durchgef"uhrten Messreihen ergaben ein koh"arentes und erwartungsgem"a\ss es Bild be\-z"ug\-lich des Stromverbrauchs. Das Skalierungsverhalten des Clusters ist wie erwartet: Der Stromverbrauch f"allt linear ab, wenn nicht an der Programmausf"uhrung beteiligte RPi-Knoten heruntergefahren und abgeschaltet werden. Ebenso ist das Skalierungsverhalten regul"ar bez"uglich der CPU-Performance bei der Ausf"uhrung von HPL. Ausf"uhrungsrate und Ausf"uhrungsdauer von HPL wachsen linear mit der Hinzunahme von Ressourcen. 

Das Skalierungsverhalten bez"uglich der Performance bei der Ausf"uhrung von STREAM f"ur $n\leq 17$ RPi-Knoten ist ebenfalls als erwartungsgem"a\ss\ zu bezeichnen. Die Ausf"uhrungsrate w"achst linear mit der Hinzunahme von Ressourcen. Die Ausf"uhrungsdauer bleibt konstant. 

F"ur $n>17$ RPi-Knoten ist das Skalierungsverhalten des Clusters irregul"ar bez"uglich der Performance bei der Ausf"uhrung von STREAM. Die Bandbreiten der Hauptspeicherzugriffe wachsen linear, doch die Ausf"uhrungsdauer nimmt deutlich zu. 

\noindent
Wie Longbottom im obigen Zitat schreibt, k"onnen auf dem RPi f"ur die CPU-Performance "ahnliche Ergebnisse erzielt werden wie auf fr"uheren Supercomputern. "Ahnliches gilt f"ur den Durchsatz an Hauptspeicherzugriffen (vgl. \url{http://www.cs.virginia.edu/stream/stream_mail/2012/0002.html}). Die vorliegenden Untersuchung erbringt den Nachweis, dass sich die verwendeten HPC-Benchmarks auch auf einem RPi-Cluster mit sinnvollen Ergebnissen ausf"uhren lassen. 

Schwierigkeiten zeigten sich bei der vorhandenen Cluster-Infrastruktur, sowohl Hardware als auch IP/SSH-Kommunikation betreffend. M"ogliche zuk"unftige Arbeiten umfassen daher zwei Felder: Optimierung der Cluster-Architektur und Erweiterungen des Versuchsaufbaus.

Die physische Architektur des Bramble erwies sich in Teilen als Hindernis. Der Aufbau ist so beengt, dass die st"andig ben"otigten Ethernet- und Mini-USB-Anschl"usse von \texttt{pi11} -- \texttt{pi20} nur schwer zug"anglich sind. Manche Mini-USB-Kabel sind so kurz abgemessen, dass sie kaum bis zum entsprechenden Knoten reichen. Ein h"aufiges Trennen der Kabel, wie in den ExperimentSuites zum Messen des Stromverbrauchs vorgesehen, wird dadurch erschwert. Zudem besteht die Gefahr, die "ubrige Hardware zu besch"adigen. 

Die einzige M"oglichkeit zum Reboot eines RPi ist Trennen und Wiederherstellen der Stromversorgung. Eine unterbrochene Netzwerkverbindung kann beim verwendeten Cluster nur durch Trennen und Wiederherstellen des Netzwerkkabels wieder hergestellt werden. Der direkte Zugriff auf einen RPi mit Tastatur und Monitor ist im Fehlerfall nicht hilfreich, da alle Einstellungen vom Bramble-Server aus vorgenommen werden. Abhilfe lie\ss e sich u.U. durch den Einbau von Reset-Kn"opfen auf den einzelnen RPi-Nodes schaffen (vgl. \url{http://raspi.tv/2012/making-a-reset-switch-for-your-rev-2-raspberry-pi}). Auch die Platzierung in einem aufrecht stehenden Gestell statt des liegenden Metallgeh"auses w"are sinnvoll (vgl. \cite{kie01} und \cite{cox13}). Damit w"aren die h"aufig ben"otigten Anschl"usse besser zug"anglich. 

F"ur zuk"unftige Experimente mit HPC-Anwendungen ist anzunehmen, dass diese die RPi-Hardware unter Umst"anden an ihre physischen Grenzen bringen. Beim vorliegenden Versuchsaufbau wurden mehrere SD-Karten durch best"andige Schreibzugriffe unbrauchbar. Hier\-f"ur sollte eine bessere L"osung gefunden werden. 

Das abweichende Skalierungsverhalten des Clusters f"ur die parallele Ausf"uhrung von\newline STREAM auf $n>17$ RPi-Knoten konnte trotz intensiver Fehlersuche nicht abschlie\ss end gekl"art werden. Hier w"are eine weiter gehende Untersuchung m"oglich. F"ur Whetstone liegt bisher keine MPICH-Implementierung vor. Auch hier besteht weiterer Forschungsbedarf, ebenso f"ur die Ausf"uhrung weiterer, bisher nicht betrachteter HPC-Benchmarks. 

Hinsichtlich des Stromverbrauchs und der CPU-Performance sind zuk"unftige Experimente mit "uber- und untertakteten CPUs der RPi-Knoten denkbar. 

Die wichtigsten neuen Erkenntnisse sind jedoch zu erwarten, wenn letztlich ein quelloffener Treiber f"ur die RPi-GPU vorliegt. Nachdem Broadcom vor Kurzem die vollst"andige Spezifikation der GPU ver"offentlicht hat (vgl. \url{http://blog.broadcom.com/chip-design/android-for-all-broadcom-gives-developers-keys-to-the-videocore-kingdom}), hat die Raspberry Pi Foundation einen entsprechenden Wettbewerb ausgeschrieben (vgl. \url{http://www.raspberrypi.org/competition-rules}). Dessen Ergebnisse und zuk"unftige Einsatz\-m"oglichkeiten der bisher kaum modifizierbaren GPU, die deutlich leistungsf"ahiger ist als die hier schwerpunktm"a\ss ig untersuchte CPU, d"urfen mit Spannung erwartet werden. 
\endinput