\chapter{Zusammenfassung und Ausblick}\label{Kap5}
\begin{quote}
\onehalfspacing
I have converted my Classic Benchmarks to run on the Linux based Raspberry Pi. These are Whetstone, [...], Linpack and Livermore Loops. [...] The Livermore Loops benchmark was used to accept the first supercomputer. So the main bragging rights are:

In 1978, the Cray 1 supercomputer cost \$7 Million, weighed 10,500 pounds and had a 115 kilowatt power supply. It was, by far, the fastest computer in the world. The Raspberry Pi costs around \$70 (CPU board, case, power supply, SD card), weighs a few ounces, uses a 5 watt power supply and is more than 4.5 times faster than the Cray 1. 

My bragging rights are that I developed and ran benchmarks, including Whetstones, on Serial 1 Cray 1\footnote{Quelle: \url{http://www.raspberrypi.org/forum/viewtopic.php?f=31&t=44080}.}.
\end{quote}

In der vorliegenden Arbeit wurde versucht, HPC-Benchmarks auf einem RPi-Cluster lauff"ahig zu machen und zu evaluieren. Dieser Proof of Concept war erfolgreich: 

Die ausgew"ahlten HPC-Benchmarks konnten mit Ausnahme von Whetstone nach dem Ausschluss vorhandener St"orfaktoren auf dem Bramble ausgef"uhrt werden. Die ExperimentSuite konnte mit kleinen "Anderungen am bestehenden Datenbank-Schema in den "ubergeordneten Versuchsaufbau integeriert werden. Die erzielten Messwerte ergaben ein koh"arentes Bild, wurden denen eines RPi-Einzelrechners gegen"ubergestellt sowie in den gr"o\ss eren Kontext der Top500-Rankings und der bisher erzielten Messwerte der Benchmark-Autoren eingeordnet. 

Der Versuchsaufbau reiht sich damit Bestrebungen der letzten Monate, einen Beowulf-Cluster aus Raspberry Pi-Einzelrechnern f"ur verteilte Berechnungen heranzuziehen. Wie Longbottom im obigen Zitat und die Projektleiter der Bramble-Projekte schreiben: Der Raspberry Pi ist in die Welt der Cluster und (zumindest ehemaligen) Supercomputer eingetreten\footnote{Vgl. auch \url{http://www.cs.virginia.edu/stream/stream_mail/2012/0002.html}.} und erzielt in den betrachteten Kategorien CPU-Performance und Speicher-Band\-breite durchaus respektable Ergebnisse. 

Schwierigkeiten zeigten sich vor allem in der vorhandenen Bramble-Infrastruktur, sowohl Hardware als auch IP/SSH-Kommunikation betreffend. M"ogliche zuk"unftige Arbeiten umfassen daher zwei Felder: 


Die physische Architektur des Bramble erwies sich in Teilen als Hindernis. Der Aufbau ist so beengt, dass die st"andig ben"otigten Ethernet- und Mini-USB-Anschl"usse von \texttt{pi11} -- \texttt{pi20} nur schwer zug"anglich sind. Manche Mini-USB-Kabel sind so kurz abgemessen, dass sie kaum bis zum entsprechenden Knoten reichen. Ein h"aufiges Trennen der Kabel, wie in den ExperimentSuites zum Messen des Stromverbrauchs einmal pro RPi-Knoten und Benchmark vorgesehen, wird dadurch erschwert.


Der vorhandene Bramble ist deutlich verbesserungsf"ahig. Vor allem die Hardware zeigt Schw"achen, nicht nur die Stromversorgung betreffend. Der physische Aufbau ist extrem beengt, sodass Ziehen und erneutes Einstecken von Mini-USB- und Netzwerkkabeln nur eingeschr"ankt m"oglich ist und die Gefahr besteht, die "ubrige Hardware dabei zu besch"adigen. Da die Unterbrechung der Stromversorgung die einzige M"oglichkeit zum Reboot eines RPi ist und eine unterbrochene Netzwerkverbindung nur durch Triggern des Netzwerkkabels wieder hergestellt werden kann, ist dieser Punkt systemkritisch f"ur zuk"unftige Untersuchungen. Abhilfe lie\ss e sich u.U. durch den Einbau von Reset-Kn"opfen auf den einzelnen RPi-Nodes schaffen\footnote{Vgl. z.B. \url{http://raspi.tv/2012/making-a-reset-switch-for-your-rev-2-raspberry-pi}.}. Auch die Platzierung der RPi-Nodes in einem aufrecht stehenden Rack statt eines liegenden Metallgeh"auses w"are von Vorteil\footnote{Vgl. \cite{kie01} und \cite{cox13}.}, damit die h"aufig ben"otigten Anschl"usse besser zug"anglich sind. 

In einem gr"osseren Rahmen w"are es interessant, fr"uher oder sp"ater auf eine MPI-Implemen\-tierung von Whetstone f"ur Raspbian zur"uckgreifen zu k"onnen. Auch die Evaluierung weiterer, bisher nicht betrachteter HPC-Benchmarks steht noch aus. Die wichtigsten Erkenntnisgewinne sind jedoch zu erwarten, wenn in naher Zukunft hoffentlich ein offener Treiber f"ur die RPi-GPU vorliegt. Nachdem Broadcom vor wenigen Wochen erstmals die vollst"andige Spezifikation der GPU zur Verf"ugung gestellt hat\footnote{Vgl. \url{http://blog.broadcom.com/chip-design/android-for-all-broadcom-gives-developers-keys-\-to-the-videocore-kingdom/}.}, hat die Raspberry Pi Foundation einen entsprechenden Wettbewerb ausgeschrieben\footnote{Vgl. \url{http://www.raspberrypi.org/competition-rules}.}. Die Ergebnisse und neuen Einsatzm"oglichkeiten der bisher kaum anprogrammierbaren GPU, die deutlich leistungsf"ahiger ist als die hier schwerpunktm"a\ss ig untersuchte CPU, d"urfen mit Spannung erwartet werden. 
\endinput