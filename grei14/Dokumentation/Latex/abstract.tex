\vspace*{2cm}

\begin{center}
    \textbf{Abstract}
\end{center}

\vspace*{1cm}

\noindent 
Seit dem Beginn seiner Entwicklung punktet der Mini-Computer Raspberry Pi durch Flexibilit"at, Preis-Leistungs-Verh"altnis niedrigschwelligen Zugang und geringen Stromverbrauch. Das macht ihn zum idealen Kandidaten f"ur einen Beowulf-Cluster. Er kann z.B. an Unversit"aten zur Forschungszwecken eingesetzt werden oder in eingeschr"anktem Rahmen einen Supercomputer simulieren. 

Tritt der Raspberry Pi in die Welt der Supercomputer ein, muss er sich auch mit ihren Spielregeln messen lassen. Die vorliegende Arbeit untersucht das Skalierungsverhalten eines Raspberry Pi-Clusters unter der Workload von Linpack, Whetstone und STREAM. Sie zeigt auf, ob und in welcher Form sich die ausgew"ahlten HPC-Benchmarks auf dem Cluster auf"uhren lassen und evaluiert die Ergebnisse. Ein Schwerpunkt liegt dabei auf dem Energieverbrauch des Clusters bei unterschiedlichen Versuchsaufbauten. 
% TODO: Ergebnisse  
