\vspace*{2cm}

\begin{center}
    \textbf{Abstract}
\end{center}

\vspace*{1cm}

\noindent 
Seit dem Beginn seiner Entwicklung punktet der Mini-Computer Raspberry Pi durch Flexibilit"at, Preis-Leistungs-Verh"altnis niedrigschwelligen Zugang und geringen Stromverbrauch. Das macht ihn zum idealen Kandidaten f"ur einen Beowulf-Cluster, mit dem z.B. ein Supercomputer simuliert werden kann. 

Tritt der Raspberry Pi in die Welt der Supercomputer ein, muss er sich auch an Hand ihrer Spielregeln messen lassen. Die vorliegende Arbeit untersucht das Skalierungsverhalten eines Raspberry Pi-Clusters unter der Workload der HPC-Benchmarks Linpack, Whetstone und STREAM. Sie zeigt, ob und in welcher Form sich die Benchmarks auf dem Raspberry Pi-Cluster auf"uhren lassen und welche Ergebnisse dabei erzielt werden. Inbesondere an Hand der Benchmarks HPLinpack und STREAM gezeigt werden, dass das Herunterfahren gerade nicht aktiver RPi-Nodes keine signifikanten Auswirkungen auf CPU-Performance und Speicher-Bandbreite hat.

Der Versuchsaufbau dient auch als Proof of Concept einer vorgegebenen Infrastruktur. Messdaten und Konfigurationen wurden in diese integriert, das Datenbankschema w"ahrend der praktischen Arbeit angepasst. Ein besonderes Augenmerk lag auf St"orf"allen des bestehenden Raspberry Pi-Clusters und L"osungsans"atzen. Eine Gegen"uberstellung der Ergebnisse mit denen eines RPi-Einzelrechners, den Messdaten der Benchmark-Autoren und eine fiktive Einordnung des Clusters in die Top500 runden die Darstellung ab.  

