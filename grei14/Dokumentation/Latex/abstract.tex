\vspace*{2cm}

\begin{center}
    \textbf{Abstract}
\end{center}

\vspace*{1cm}

\noindent 
Seit dem Beginn seiner Entwicklung punktet der Mini-Computer Raspberry Pi durch Flexibilit"at, Preis-Leistungs-Verh"altnis, niedrigschwelligen Zugang und geringen Stromverbrauch. Das macht ihn zum idealen Kandidaten f"ur einen Beowulf-Cluster. Er kann z.B. an Unversit"aten zur Forschungszwecken eingesetzt werden oder in eingeschr"anktem Rahmen einen Supercomputer simulieren. 

Tritt der Raspberry Pi in die Welt der Supercomputer ein, muss er sich auch mit ihren Spielregeln messen lassen. Die vorliegende Arbeit untersucht das Skalierungsverhalten eines Raspberry Pi-Clusters mit 20 Raspberry Pi-Knoten bei der parallen Ausf"uhrung der HPC-Benchmarks HPL und STREAM. Als Parameter werden Performance (Ausf"uhrungsrate und Ausf"uhrungsdauer) und Energieverbrauch betrachtet. 

Die Ergebnisse zeigen ein koh"arentes und erwartungsgem"a\ss es Skalierungsverhalten des Clusters bez"uglich Stromverbrauch, Performance bei der Ausf"uhrung von HPL und Performance bei der Ausf"uhrung von STREAM auf bis zu 17 Raspberry Pi-Knoten. Der Versuchsaufbau ist weiterer Schritt auf dem Weg der Ausf"uhrung von parallelen und HPC-Anwendungen auf dem Raspberry Pi und erweitert dessen Einsatzm"oglichkeiten. 