\vspace*{2cm}

\begin{center}
    \textbf{Abstract}
\end{center}

\vspace*{1cm}

\noindent 
Seit Beginn seiner Entwicklung punktet der Mini-Computer Raspberry Pi durch Flexibilit"at, Preis-Leistungs-Verh"altnis, niedrigschwelligen Zugang und geringen Stromverbrauch. Das macht ihn zum idealen Kandidaten f"ur einen Beowulf-Cluster. Er wird z.B. an Universit"aten zu Forschungszwecken und f"ur parallele Berechnungen eingesetzt.  

Die vorliegende Arbeit untersucht das Skalierungsverhalten eines Raspberry Pi-Clusters mit 20 Raspberry Pi-Knoten unter Verwendung der HPC-Benchmarks HPL und STREAM. Als Parameter werden Performance (Ausf"uhrungsrate und Ausf"uhrungsdauer) und Energieverbrauch betrachtet. 

Die Ergebnisse zeigen ein koh"arentes und erwartungsgem"a\ss es Skalierungsverhalten des Clusters bez"uglich Stromverbrauch, Performance bei der Ausf"uhrung von HPL und Performance bei der Ausf"uhrung von STREAM auf bis zu 17 Raspberry Pi-Knoten. Die Untersuchung erbringt somit den Nachweis, dass die Ausf"uhrung der ausgew"ahlten HPC-Benchmarks auf einem Raspberry Pi-Cluster mit sinnvollen Ergebnissen m"oglich ist. Der Versuchsaufbau ist damit ein weiterer Schritt auf dem Weg der Ausf"uhrung von parallelen und HPC-Anwendungen auf dem Raspberry Pi. Er erweitert dessen Einsatzm"oglichkeiten, zeigt aber auch Grenzen auf. 