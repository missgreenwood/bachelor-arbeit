\vspace*{2cm}

\begin{center}
    \textbf{Abstract}
\end{center}

\vspace*{1cm}

\noindent
% TODO: Überarbeitung  
Seit dem Beginn seiner Entwicklung punktet der Mini-Computer Raspberry Pi durch Flexibilit"at, Preis-Leistungs-Verh"altnis, niedrigschwelligen Zugang und geringen Stromverbrauch. Das macht ihn zum idealen Kandidaten f"ur einen Beowulf-Cluster. Dieser wird z.B. zu Forschungszwecken an Universit"aten eingesetzt oder in eingeschr"anktem Rahmen, zur Simulation eines Supercomputers. 

Tritt der Raspberry Pi in die Welt der Supercomputer ein, muss er sich auch mit ihren Spielregeln messen lassen. Die vorliegende Arbeit untersucht das Skalierungsverhalten eines Raspberry Pi-Clusters unter der Workload ausgew"ahlter HPC-Benchmarks. Sie zeigt, ob und in welcher Form sich die Benchmarks auf dem Cluster auf"uhren lassen und welche Ergebnisse dabei erzielt werden. Im Zentrum steht neben der Machbarkeit der Energieverbrauch unterschiedlicher Versuchsaufbauten, der mit Hilfe eines Strommessger"ats evaluiert wird. 

% TODO: Ergebnisse  

