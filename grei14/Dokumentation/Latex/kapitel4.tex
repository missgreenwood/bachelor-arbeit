\chapter{Interpretation}\label{Kap4}

Das folgende Kapitel dient der Bewertung und Einordnung der Untersuchungsergebnisse. Im Fokus steht dabei das Skalierungsverhalten des Bramble.    

\section{HPL: Performance}\label{Interpretation-Linpack}
Wie in Kapitel \ref{Ergebnisse-HPL} dargestellt, besteht ein proportionales Verh"altnis zwischen dem Quadrat der Problemgr"o\ss e und der Gr"o\ss e des Gesamthauptspeichers. Somit ist bei Hinzunahme von Ressourcen in Form von RPi-Knoten ein linearer Anstieg der Ausf"uhrungsrate zu erwarten. Das gilt ebenfalls f"ur die Ausf"uhrungsdauer. Die Diagramme \ref{fig:hpl1} und \ref{fig:hpl2} best"atigen die Erwartungen: Der Anstieg der Ausf"uhrungsdauer ist nahezu linear und betr"agt im Mittel 0.031 GFLOPs pro zus"atzlicher Ressource. Die maximale Abweichung hiervon betr"agt 0.008 GFLOPs. Der Anstieg der Ausf"uhrungdauer ist nahezu linear und betr"agt im Mittel 13.28 s. Die maximale Abweichung hiervon betr"agt 8.24 s.

Das Skalierungsverhalten des Bramble bei der Ausf"uhrung von HPL kann somit als erwartungsgem"a\ss\ bezeichnet werden.

Versuchsaufbau und Funktionsweise des Benchmarks lassen erwarten, dass sich kein signifikanter Unterschied in Ausf"uhrungsrate und Ausf"uhrungsdauer zeigt, wenn nicht an der Programmausf"uhrung beteiligte RPi-Knoten heruntergefahren werden. Die Diagramme \ref{fig:hpl5} und \ref{fig:hpl6} best"atigen diese Erwartung: Messreihe 1 und Messreihe 2 werden nahezu identisch dargestellt. Die Ausf"uhrungsdauer in Messreihe 2 ist geringf"ugig h"oher als in Messreihe 1. Die Abweichung betr"agt im Mittel 0.41 s. Die maximale Abweichung betr"agt 0.63 s. Die Ausf"uhrungsrate ist in Messreihe 2 f"ur $n=4$ und $n=16$ RPi-Knoten geringf"ugig niedriger als in Messreihe 1. Die Abweichung betr"agt im Mittel 0.001 GFLOPs (gerundet auf drei Nachkommastellen an Hand der Messgenauigkeit, vgl. Kap. \ref{Ergebnisse-HPL}). Die maximale Abweichung betr"agt 0.001 GFLOPs. 

Es l"asst sich schlussfolgern, dass das Herunterfahren nicht mehr aktiver RPi-Knoten keine signifikanten Auswirkungen auf Ausf"uhrungsrate und Ausf"uhrungsdauer von HPL auf dem Bramble hat. Das Skalierungsverhalten in Messreihe 1 und Messreihe 2 kann als gleich bezeichnet werden. 

\section{STREAM: Performance}\label{Interpretation-Stream}

Die Funktionsweise des Benchmarks l"asst einen linearen Anstieg der Ausf"uhrungsrate bei der Hinzunahme von Ressourcen erwarten. F"ur die Ausf"uhrungsdauer auf jeder einzelnen CPU ist ein konstantes Verhalten zu erwarten. Die Diagramme \ref{fig:stream1} und \ref{fig:stream2} best"atigen die Erwartungen f"ur $n\leq 17$ RPi-Knoten: Der Anstieg der Ausf"uhrungsdauer ist nahezu linear und betr"agt im Mittel 245.9 MB/s (Copy), 205.1 MB/s (Scale), 275.5 MB/s (Add) bzw. 266.9 MB/s (Triad) pro zus"atzlicher Ressource. Die maximale Abweichung hiervon betr"agt 54.2 MB/s (Copy), 19.7 MB/s (Scale), 25.9 MB/s (Add) bzw. 25.6 MB/s (Triad). Die Ausf"uhrungsdauer verh"alt sich nahezu konstant und betr"agt im Mittel 0.118791 s (Copy), 0.156077 (Scale), 0.173474 (Add) bzw. 0.179196 (Triad). Die maximale Abweichung hiervon betr"agt 0.001525 s (Copy), 0.002114 s (Scale), 0.002463 s (Add) bzw. 0.003793 s (Triad). 

Das Skalierungsverhalten des Bramble bei der Ausf"uhrung von STREAM kann somit f"ur $n\leq 17$ RPi-Knoten als erwartungsgem"a\ss\ bezeichnet werden. 

Versuchsaufbau und Funktionsweise des Benchmarks legen nahe, dass sich kein signifikanter Unterschied in Ausf"uhrungsrate und Ausf"uhrungsdauer zeigt, wenn nicht an der Programmausf"uhrung beteiligte RPi-Knoten heruntergefahren werden. Die Diagramme \ref{fig:stream5} und \ref{fig:stream6} best"atigen diese Erwartung f"ur $n\leq 17$ RPi-Knoten: Ausf"uhrungsraten von Messreihe 1 und Messreihe 2 werden nahezu identisch dargestellt. Die Abweichung von Messreihe 2 gegen"uber Messreihe 1 betr"agt im Mittel 0.1 MB/s (Copy), 0.5 MB/s (Scale), 2.9 MB/s (Add) bzw. 2.95 MB/s (Triad). Die maximale Abweichung gegen"uber Messreihe 1 betr"agt 25.4 MB/s (Scale auf $n=16$ RPi-Knoten). Bei der Ausf"uhrungsdauer betr"agt die Abweichung gegen"uber Messreihe 1 im Mittel 0.000123 s (Copy), 0.000129 s (Scale), 0.000190 s (Add) bzw. 0.000122 s (Triad). Die maximale Abweichung gegen"uber Messreihe 1 betr"agt 0.000823 s (Copy auf $n=15$ RPi-Knoten). 

F"ur $n\leq 17$ RPi-Knoten l"asst sich schlussfolgern, dass das Herunterfahren nicht mehr aktiver RPi-Knoten keine signifikanten Auswirkungen auf Ausf"uhrungsrate und Ausf"uhrungsdauer von STREAM auf dem Bramble hat. Das Skalierungsverhalten in Messreihe 1 und Messreihe 2 kann f"ur $n\leq 17$ RPi-Knoten als gleich bezeichnet werden.

F"ur $n > 17$ RPi-Knoten weicht das Skalierungsverhalten des Bramble von den Erwartungen ab. Tabelle \ref{fig:stream-abweichung} stellt erwartete und erzielte Messwerte f"ur Messreihe 1 gegen"uber. Abweichungen von mehr als 1 MB/s (Ausf"uhrungsrate) und 0.01 s (Ausf"uhrungsdauer) gegen"uber den erwarteten Werten sind rot markiert. Die erwarteten Werte werden f"ur \textit{n} RPi-Knoten wie folgt bestimmt: 
\[\text{Erwartete Performance} = \frac{\text{Performance f"ur }n=17}{17}\ast n\] 
\[\text{Erwartete Ausf"uhrungsdauer = Durchschnittliche Ausf"uhrungsdauer f"ur }n\leq 17\]
\begin{figure}
  \centering
  \begin{tabular}{|l|c|c|c|}
    \hline 
    \textbf{Aktive RPis:} & \textbf{17} & \textbf{18} & \textbf{19}\\ 
    \hline 
    \textbf{Copy in MB/s (erwartet):} & 4556.4 & 4824.4 & 5092.5\\
    \hline 
    \textbf{Copy in MB/s (erzielt):} & 4556.4 & \textcolor{red}{3451.6} & \textcolor{red}{3642.9}\\
    \hline 
    \textbf{Copy in s (erwartet):} & 0.118791 & 0.118791 & 0.118791\\
    \hline 
    \textbf{Copy in s (erzielt):} & 0.119821 & \textcolor{red}{0.168030} & \textcolor{red}{0.167579}\\
    \hline 
    \textbf{Scale in MB/s (erwartet):} & 3497.7 & 3703.5 & 3909.1\\
    \hline 
    \textbf{Scale in MB/s (erzielt):} & 3497.7 & \textcolor{red}{3236.2} & \textcolor{red}{3421.8}\\
    \hline 
	\textbf{Scale in s (erwartet):} & 0.156077 & 0.156077 & 0.156077\\
    \hline 
    \textbf{Scale in s (erzielt):} & 0.158162 & \textcolor{red}{0.179140} & \textcolor{red}{0.178560}\\
    \hline 
    \textbf{Add in MB/s (erwartet):} & 1672.5 & 4978.6 & 5255.2\\
    \hline 
    \textbf{Add in MB/s (erzielt):} & 1672.5 & \textcolor{red}{4501.4} & \textcolor{red}{4765.8}\\
    \hline 
    \textbf{Add in s (erwartet):} & 0.173474 & 0.173474 & 0.173474\\
    \hline 
    \textbf{Add in s (erzielt):} & 0.174494 & \textcolor{red}{0.192586} & \textcolor{red}{0.192147}\\
    \hline 
    \textbf{Triad in MB/s (erwartet):} & 4557.4 & 4852.5 & 4093.6\\
    \hline 
    \textbf{Triad in MB/s (erzielt):} & 4557.4 & \textcolor{red}{4377.4} & \textcolor{red}{4632.1}\\
    \hline 
    \textbf{Triad in s (erwartet):} & 0.179196 & 0.179196 & 0.179196\\
    \hline 
    \textbf{Triad in s (erzielt):} & 0.179882 & \textcolor{red}{0.198159} & \textcolor{red}{0.197686}\\
    \hline 
  \end{tabular}
  \caption{Erwartete und erzielte Messwerte f"ur STREAM auf $n\geq 17$ RPi-Knoten.}\label{fig:stream-abweichung}
\end{figure}

\noindent
Es stellt sich die Frage, warum f"ur $n>17$ RPi-Knoten eine deutlich schlechtere Performance und verl"angerte Ausf"uhrungsdauer auftreten. Folgende Erkl"arungen sind denkbar: 
\begin{enumerate}\bfseries
	\item Funktionsweise des Benchmarks.\\
\normalfont
Wie zu Beginn des Kapitels dargestellt, ist bei der Hinzunahme von Ressourcen auf einem Rechencluster mit einem linearen Anstieg der Ausf"uhrungszeit und ann"ahernd konstanter Ausf"uhrungsdauer zu rechnen. McCalpin schreibt hierzu: 
\begin{quote}
\onehalfspacing
[\dots] unless something is very wrong, the performance of a cluster will be the performance of a node times the number of nodes (vgl. \url{http://www.cs.virginia.edu/stream/ref.html}).
\end{quote} 
Ein erw"unschter Effekt ist somit nach der Funktionsweise des Benchmarks auszuschlie\ss en. 
	\textbf{\item Architektur des Bramble.}\\
M"ogliche Ursachen sind Systemzeit, Netzwerk und Netz-Dateisystem. 

Wie in Kap. \ref{Bramble-Versuchsaufbau} dargestellt, hat der RPi keine eingebaute Systemuhr. Die Zeitsynchronisation der RPi-Knoten erfolgt einmal beim Bootvorgang gegen"uber dem Open\-NTP-Server auf \texttt{careme} (vgl. \cite{kli13}). Wenn Rechenlast ungleich auf die RPi-Knoten verteilt wird und die beteiligten CPUs sehr ausgelastet sind, w"are es denkbar, dass die Systemzeit dieser Knoten driftet und zu abweichenden Messergebnissen bei der Ausf"uhrungsdauer f"uhrt. Dieser Effekt kann hier ausgeschlossen werden: Die Rechenlast ist bei $n\leq 17$ Knoten nicht weniger ungleich verteilt ist als bei $n>17$ Knoten, sodass der Effekt schon fr"uher eintreten m"usste. Die Rechenlast der parallelen Ausf"uhrung von STREAM ist zudem f"ur alle beteiligten Knoten gleich. 

Wie in Kap. \ref{Bramble-Spezi} dargestellt, sind Server und RPi-Knoten "uber ein Ethernet-Netzwerk verbunden. Wie bei \cite{kli13} erl"autert, kann damit ein maximaler Datendurchsatz von ca. 2 MB/s erreicht werden. Ein Erkl"arungsansatz war das "Uberschreiten dieser Obergrenze bei mehr als 17 parallelen Ausf"uhrungen. 

Bez"uglich der Bandbreiten zeigte sich durch Nachrechnen eine Steigerung um rund 35 MB  f"ur Copy und Scale pro zus"atzlichem RPi-Knoten. F"ur Add und Triad erfolgt eine Steigerung um rund 50 MB. Hierbei handelt es sich nat"urlich nicht um den Durchsatz des Netzwerks, sondern um den Durchsatz von Hauptspeicherzugriffen jedes einzelnen RPi-Knotens. Diese Steigerung erfolgt auch bei $n>17$
RPi-Knoten. Der Durchsatz von Hauptspeicherzugriffen verh"alt sich somit erwartungsgem"a\ss\ f"ur alle \textit{n}. Ein "Uberschreiten des maximalen Netzwerk-Datendurchsatzes scheidet somit als Ursache aus. 

Binaries und verwendete Bibliotheken der Benchmarks liegen im geteilten Verzeichnis \texttt{/srv}. Es erschien denkbar, dass mehr als 17 parallele Zugriffe darauf das Netz-Dateisystem "uberlasten. Deswegen wurde probeweise auf den SD-Karten aller RPi-Knoten eine neue Partition erstellt und der Inhalt des geteilten Verzeichnisses hinein kopiert. In der Datei \texttt{/etc/fstab} aller RPi-Knoten wurde der Mountpoint f"ur \texttt{/srv} tempor"ar entsprechend ge"andert. Das Resultat waren Performance-Einbr"uche bei STREAM schon ab $n=13$ RPi-Knoten, womit das Netz-Dateisystem als Ursache ausscheidet.
\textbf{\item Ausf"uhrung des Benchmarks.}\\
Eine m"ogliche Ursache ist die Funktionsweise von MPICH.

Wie in Kap. \ref{Versuchsaufbau} und \ref{Bramble-Versuchsaufbau} beschrieben, wird die parallele Ausf"uhrung von STREAM durch MPICH angesto\ss en, das den im Machinefile spezifizierten CPUs eine bestimmte Anzahl an parallelen Programmaufrufen zuweist. Ein weiterer Erkl"arungsansatz war die "Uberlastung des Netzwerks durch den Kommunikations-Overhead ab einem Schwellenwert von $n=18$ parallelen Programmaufrufen. 

MPI nutzt zur Interprozesskommunikation u.a. Broadcast- und Unicast-Nachrichten. Bei der parallelen Ausf"uhrung von Anwendungen werden meist die Funktionen \texttt{MPI\_Bcast} und \texttt{MPI\_Gather} aufgerufen. Hierbei schickt ein Root-Prozess entweder eine Nachricht an alle beteiligten CPUs bzw. Prozesse oder sammelt die Daten aller beteiligten CPUs bzw. Prozesse ein (vgl. \cite{pie12}). Es erschien denkbar, dass eine Broadcast-Nachricht an mehr als 17 Prozessoren das Netzwerk "uberlastet und zu den beobachteten Performance-Einbr"uchen f"uhrt. Daher wurde der Quellcode von STREAM testweise so ver"andert, dass jedes Modul 100 Mal statt zehn Mal ausgef"uhrt wird\footnote{Dazu muss die Variable \texttt{ntimes} ver"andert werden, vgl. \url{http://www.cs.virginia.edu/stream/ref.html}.}. Bei einer l"angeren Ausf"uhrungsdauer des Programms war zu erwarten, dass der Kommunikations-Overhead von MPICH abnimmt und erwartungsgem"a\ss e Performance-Resultate erzielt werden. Das war nicht der Fall. Folgende Ergebnisse wurden f"ur $n=18$ RPi-Knoten erzielt: 
\begin{figure}
  \centering
  \begin{tabular}{|l|c|}
    \hline 
    \textbf{Aktive RPi-Knoten:} & \textbf{18}\\ 
    \hline 
    \textbf{Copy in MB/s (erwartet):} & 4824.4\\
    \hline 
    \textbf{Copy in MB/s (erzielt):} & \textcolor{red}{3398.3}\\
    \hline 
    \textbf{Copy in s (erwartet):} & 0.118791\\
    \hline 
    \textbf{Copy in s (erzielt):} & \textcolor{red}{0.170848}\\
    \hline 
    \textbf{Scale in MB/s (erwartet):} & 3703.5\\
    \hline 
    \textbf{Scale in MB/s (erzielt):} & \textcolor{red}{3189.5}\\
    \hline 
	\textbf{Scale in s (erwartet):} & 0.156077\\
    \hline 
    \textbf{Scale in s (erzielt):} & \textcolor{red}{0.181711}\\
    \hline 
    \textbf{Add in MB/s (erwartet):} & 4978.6\\
    \hline 
    \textbf{Add in MB/s (erzielt):}& \textcolor{red}{4373.3}\\
    \hline 
    \textbf{Add in s (erwartet):} & 0.173474\\
    \hline 
    \textbf{Add in s (erzielt):} & \textcolor{red}{0.199341}\\
    \hline 
    \textbf{Triad in MB/s (erwartet):} & 4852.5\\
    \hline 
    \textbf{Triad in MB/s (erzielt):} & \textcolor{red}{4149.8}\\
    \hline 
    \textbf{Triad in s (erwartet):} & 0.179196\\
    \hline 
    \textbf{Triad in s (erzielt):} & \textcolor{red}{0.209507}\\
    \hline 
  \end{tabular}
  \caption{Erwartete und erzielte Messwerte f"ur STREAM auf $n=18$ RPi-Knoten mit \texttt{ntimes=100}.}\label{fig:stream-ntimes100}
\end{figure}
\noindent
Die Resultate fallen noch schlechter aus als bei STREAM auf $n=18$ RPi-Knoten mit \texttt{ntimes=10} (vgl. Tabelle \ref{fig:stream-abweichung}). Damit scheidet ein Kommunikations-Overhead von MPICH als Erkl"arung aus. Die Ursache konnte somit nicht zweifelsfrei gekl"art werden. 
\end{enumerate}

\section{Stromverbrauch}
Ziel der Strommessung war die Ermittlung des Skalierungsverhaltens des Bramble bez"uglich des Stromverbrauchs. In Messreihe 1 werden laufend Ressourcen in Form von RPi-Knoten hinzugenommen. Auch nicht aktive RPi-Knoten sind angeschaltet, nehmen somit Strom auf. In Messreihe 2 werden nicht aktive RPi-Knoten abgeschaltet, nehmen somit keinen Strom auf. 

Es ist zu erwarten, dass der Stromverbrauch deutlich sinkt, wenn nicht aktive RPi-Knoten von der Stromversorgung getrennt werden. Zur "Uberpr"ufung dieser Erwartung wurden Stromverbrauch bei der Ausf"uhrung von HPL und STREAM (Messreihe 1 und Messreihe 2), Zuwachs pro aktivem RPi-Knoten (Messreihe 1) und Zuwachs pro angeschaltetem und aktivem RPi-Knoten (Messreihe 2) empirisch ermittelt. 

Die Ergebnisse werden in Tabelle \ref{fig:stromvergleich} als Mittelwerte dargestellt (gerundet auf 4 Stellen ohne f"uhrende Nullen an Hand der Messgenauigkeit, vgl. Kap. \ref{Strommessung}). 
\begin{figure}
  \centering
  \begin{tabular}{|l|c|c|c|c|}
    \hline 
    & HPL (1) & HPL (2) & STREAM (1) & STREAM (2)\\ 
    \hline 
	\textbf{ExperimentSuite} & 110 W & 88 & 111 W & 94 W\\
    \hline 
    \textbf{Maximale Abweichung} & 4 W & 3 W & 5 W & 5 W\\
	\hline
    \textbf{Zuwachs pro RPi-Knoten} & 0 W & 8 W & 0 W & 2 W\\
    \hline 
    \textbf{Messreihe} & 111 W & 88 W & 111 W & 94 W\\
    \hline 
  \end{tabular}
  \caption{Stromverbrauch des Bramble, Messreihe 1 und Messreihe 2.}
\label{fig:stromvergleich}
\end{figure}
\noindent

Die Messergebnisse best"atigen die Erwartung, dass der Stromverbrauch des Bramble pro abgeschaltetem RPi-Knoten deutlich abnimmt: Im Mittel um 8 W pro RPi-Knoten bei der Ausf"uhrung von HPL und um 2 W pro RPi-Knoten bei der Ausf"uhrung von STREAM. Die Stromverbrauch bei der Durchf"uhrung von Messreihe 2 ist damit im Mittel um 23 W (HPL) bzw. 17 W (STREAM) niedriger als in Messreihe 1. Das Skalierungsverhalten des Bramble bez"uglich des Stromverbrauchs kann somit als erwartungsgem"a\ss\ bezeichnet werden. 
\endinput 