\chapter{Interpretation}\label{Kap4}

Das folgende Kapitel dient der Bewertung und Einordnung der ermittelten Ergebnisse: Performance f"ur HPLinpack und STREAM f"ur Messreihe 1 gegen"uber Messreihe 2 sowie Stromverbrauch in Messreihe 1 gegen"uber Messreihe 2. Dabei ist das Skalierungsverhalten des Bramble bez"uglich Performance und Stromverbrauch ausschlaggebend. Abschlie\ss end werden Grenzen des Versuchsaufbaus diskutiert.   

\section{HPLinpack: Performance}\label{Interpretation-Linpack}

% TODO: Interpretation HPL Messreihe 1 (Skalierungsverhalten!)

%Der Versuchsaufbau und die Steuerung der verteilten Ausf"uhrung der Benchmark-Programme durch MPICH lassen vermuten, dass sich kein signifikanter Unterschied in der Performance zeigt, je nachdem, ob nicht mehr an der Programmausf"uhrung beteiligte RPi-Knoten heruntergefahren werden oder nicht. Wie in Kap. \ref{Bramble-Versuchsaufbau} dargestellt, ist im Machinefile genau festgelegt, welchen CPUs bzw. RPi-Knoten mit mpiexec eine die Ausf"uhrung des Benchmark-Programms zugewiesen wird und in welcher Reihenfolge sie f"ur die Ausf"uhrung herangezogen werden.
%
%Die Diagramme \ref{fig:hpl5} und \ref{fig:hpl6} best"atigen diese Erwartung: Messreihe 1 und Messreihe 2 werden in den Grafiken nahezu identisch dargestellt. Tabelle \ref{fig:hpl-vgl} stellt die minimalen, maximalen und durchschnittlichen Werte f"ur Ausf"uhrungsrate und Ausf"uhrungsdauer beider Messreihen gegen"uber: 
%%TODO: Werte ermitteln und eintragen
%%\begin{figure}
%%  \centering
%%  \begin{tabular}{|l|c|c|}
%%    \hline 
%%     & \textbf{20 RPis angeschaltet} & \textbf{n+1 RPis angeschaltet}\\ 
%%    \hline 
%%    \textbf{Ausf"uhrungsrate/Min.} & GFLOPs & GFLOPs\\
%%    \hline 
%%    \textbf{Ausf"uhrungsrate/Max.} & GFLOPs & GFLOPs\\
%%    \hline 
%%    \textbf{Ausf"uhrungsrate/$\varnothing$} & GFLOPs & GFLOPs\\
%%    \hline 
%%    \textbf{Ausf"uhrungsdauer/Min.} &  s & s\\
%%    \hline 
%%    \textbf{Ausf"uhrungsdauer/Max.} &  s & s\\
%%	\hline 
%%    \textbf{Ausf"uhrungsdauer/$\varnothing$} & s & s\\
%%    \hline
%%  \end{tabular}
%%  \caption{Performance f"ur HPLinpack, Messreihe 1 und Messreihe 2.}\label{fig:hpl-vgl}
%%\end{figure}
%\noindent
%
%% TODO: genaue Werte!
%Dabei "uberschreiten die Differenzen in keinem Fall die Gr"o\ss enordnung von %z
%, liegen also innerhalb des Messfehlers (vgl. Kap. \ref{Ergebnisse-HPL}). Somit l"asst sich schlussfolgern, dass das Herunterfahren nicht mehr aktiver RPi-Knoten keine signifikanten Auswirkungen auf die Performance von HPLinpack auf dem Bramble hat. Das Skalierungsverhalten bez"uglich der Performance ist somit f"ur beide Messreihen gleich. 

\section{STREAM: Performance}\label{Interpretation-Stream}

Die Konzeption des STREAM-Benchmarks l"asst erwarten, dass das Hinzunehmen von Ressourcen in Form von RPi-Knoten einen linearen Anstieg der Ausf"uhrungsrate zu Folge hat. Der Autor des Benchmarks schreibt hierzu: 
\begin{quote}
\onehalfspacing
[\dots] unless something is very wrong, the performance of a cluster will be the performance of a node times the number of nodes (vgl. \url{http://www.cs.virginia.edu/stream/ref.html}).
\end{quote}
Die Ausf"uhrungsdauer auf jeder einzelnen CPU sollte somit ann"ahernd konstant bleiben, auch wenn Ressourcen hinzugenommen werden.

Diagramme \ref{fig:stream1} und \ref{fig:stream2} best"atigen dieses Verhalten f"ur die parallele Ausf"uhrung von STREAM auf $n\leq 17$ RPi-Knoten: Der Anstieg der  Ausf"uhrungsrate bei Hinzunahme von RPi-Knoten ist nahezu linear. Geringe Abweichungen zeigen sich bei der Ausf"uhrungsdauer: Sie wird in Mikrosekunden gemessen. Die obere und untere maximale Abweichung hiervon betr"agt 0.001455 bzw. -0.001525 s f"ur Copy, . % TODO Grenze bestimmen

Das Skalierungsverhalten des Bramble bei der parallelen Ausf"uhrung von STREAM auf $n\leq 17$ RPi-Knoten kann somit als erwartungsgem"a\ss\ bezeichnet werden. 

Der Funktionsweise des Benchmarks zu Folge ist zu erwarten, dass auch das Herunterfahren nicht mehr aktiver RPi-Knoten keinen signifikanten Effekt auf Performance und Ausf"uhrungsdauer der anderen RPi-Knoten hat. Die Diagramme \ref{fig:stream5} und \ref{fig:stream6} best"atigen dieses Verhalten f"ur $n\leq 17$ RPi-Knoten: Die Ausf"uhrungsrate ist in Messreihe 2 nahezu identisch mit der Ausf"uhrungsrate in Messreihe 1. Bei der durchschnittlichen Ausf"uhrungsdauer von Messreihe 1 gegen"uber Messreihe 2 zeigen sich, wie schon innerhalb von Messreihe 1, leichte Unterschiede. Hier liegt die Abweichung von Messreihe 2 gegen"uber Messreihe 1 bei maximal 0.000823 s (Ausf"uhrungsrate des Moduls Copy auf $n=15$ RPi-Knoten). Die Abweichung gegen"uber der durchschnittlichen Ausf"uhrungsdauer liegt f"ur Messreihe 1 bei 0.000968 s, f"ur Messreihe 2 bei -0.000145 s. % TODO Grenze bestimmen

Das Skalierungsverhalten des Bramble kann somit auch f"ur Messreihe 2 mit $n\leq 17$ RPi-Knoten als erwartungsgem"a\ss\ bezeichnet werden.  
\noindent
F"ur $n > 17$ RPi-Knoten zeigt der Bramble ein abweichendes Skalierungsverhalten bez"uglich Ausf"uhrungsrate und Ausf"uhrungsdauer. Tabelle \ref{fig:stream-abweichung} stellt erwartete und erzielte Messwerte f"ur Messreihe 1 gegen"uber. Abweichungen von mehr als 1 MB/s (Performance) und 0.01 s (Ausf"uhrungsdauer) gegen"uber den erwarteten Werten sind darin rot markiert. Die erwarteten Werte werden f"ur \textit{n} RPi-Knoten wie folgt bestimmt: 
\[\text{Erwartete Performance} = \frac{\text{Performance f"ur }n=17}{17}\ast n\] 
\[\text{Erwartete Ausf"uhrungsdauer = Durchschnittliche Ausf"uhrungsdauer f"ur }n\leq 17\]
\begin{figure}
  \centering
  \begin{tabular}{|l|c|c|c|}
    \hline 
    \textbf{Aktive RPi-Knoten:} & \textbf{17} & \textbf{18} & \textbf{19}\\ 
    \hline 
    \textbf{Copy in MB/s (erwartet):} & 4556.4 & 4824.4 & 5092.5\\
    \hline 
    \textbf{Copy in MB/s (erzielt):} & 4556.4 & \textcolor{red}{3451.6} & \textcolor{red}{3642.9}\\
    \hline 
    \textbf{Copy in s (erwartet):} & 0.118791 & 0.118791 & 0.118791\\
    \hline 
    \textbf{Copy in s (erzielt):} & 0.119821 & \textcolor{red}{0.168030} & \textcolor{red}{0.167579}\\
    \hline 
    \textbf{Scale in MB/s (erwartet):} & 3497.7 & 3703.5 & 3909.1\\
    \hline 
    \textbf{Scale in MB/s (erzielt):} & 3497.7 & \textcolor{red}{3236.2} & \textcolor{red}{3421.8}\\
    \hline 
	\textbf{Scale in s (erwartet):} & 0.156077 & 0.156077 & 0.156077\\
    \hline 
    \textbf{Scale in s (erzielt):} & 0.158162 & \textcolor{red}{0.179140} & \textcolor{red}{0.178560}\\
    \hline 
    \textbf{Add in MB/s (erwartet):} & 1672.5 & 4978.6 & 5255.2\\
    \hline 
    \textbf{Add in MB/s (erzielt):} & 1672.5 & \textcolor{red}{4501.4} & \textcolor{red}{4765.8}\\
    \hline 
    \textbf{Add in s (erwartet):} & 0.173474 & 0.173474 & 0.173474\\
    \hline 
    \textbf{Add in s (erzielt):} & 0.174494 & \textcolor{red}{0.192586} & \textcolor{red}{0.192147}\\
    \hline 
    \textbf{Triad in MB/s (erwartet):} & 4557.4 & 4852.5 & 4093.6\\
    \hline 
    \textbf{Triad in MB/s (erzielt):} & 4557.4 & \textcolor{red}{4377.4} & \textcolor{red}{4632.1}\\
    \hline 
    \textbf{Triad in s (erwartet):} & 0.179196 & 0.179196 & 0.179196\\
    \hline 
    \textbf{Triad in s (erzielt):} & 0.179882 & \textcolor{red}{0.198159} & \textcolor{red}{0.197686}\\
    \hline 
  \end{tabular}
  \caption{Erwartete und erzielte Messwerte f"ur STREAM auf $n\geq 17$ RPi-Knoten.}\label{fig:stream-abweichung}
\end{figure}
\noindent
Es stellt sich die Frage, warum das Skalierungsverhalten des Bramble f"ur $n\leq 17$ RPi-Knoten als erwartungsgem"a\ss\ zu bezeichnen ist und ab $n>17$ RPi-Knoten eine deutlich schlechtere Performance und verl"angerte Ausf"uhrungsdauer aufweist. Folgende Erkl"arungsversuche sind denkbar: 
\begin{enumerate}\bfseries
	\item Funktionsweise des Benchmarks.\\
\normalfont
Wie zu Beginn des Kapitels dargestellt, ist auf einem Rechencluster mit einem linearen Anstieg der Performance und ann"ahernd konstanter Ausf"uhrungsdauer bei der Hinzunahme von Ressourcen zu rechnen. Dass hier ein gew"unschter bzw. erwartungsgem"a\ss er Effekt vorliegt, ist nach der Konzeption des Benchmarks auszuschlie\ss en. 
	\textbf{\item Architektur des Bramble.}\\
Hier sind zwei Ursachen denkbar: Systemzeit und Netzwerk. 

Wie in Kap. \ref{Bramble-Versuchsaufbau} dargestellt, verf"ugt der RPi "uber keine eingebaute Systemuhr. Die Zeitsynchronisation muss daher anders erfolgen, in diesem Fall durch die Synchronisation gegen einen NTP-Server auf \texttt{careme} beim Bootvorgang. Wenn ungleich Rechenlast auf die RPi-Knoten verteilt wird, so w"are es denkbar, dass die Systemzeit dieser Knoten driftet und zu abweichenden Messergebnissen in der Ausf"uhrungsdauer von STREAM f"uhrt. Der Effekt kann in diesem Fall ausgeschlossen werden, da die Last bei $n\leq 17$ Knoten nicht weniger ungleich verteilt ist als bei $n \geq 18$ Knoten, sodass der beschriebene Effekt schon fr"uher eintreten m"usste. 

Wie in Kap. \ref{Bramble-Spezi} dargestellt, kommunizieren Bramble-Server und RPi-Knoten "uber ein Ethernet-Netzwerk. Wie bei \cite{kli13} erl"autert, kann damit ein maximaler Datendurchsatz von ca. 2 MB/s erreicht werden. Ein Erkl"arungsansatz f"ur das abweichende Skalierungsverhalten war daher das Erreichen dieser Obergrenze. 

Bez"uglich der Bandbreiten zeigte sich durch Nachrechnen, dass bei den Modulen Copy und Scale eine Steigerung um ca. 30-40 MB, bei Add und Triad um ca. 50 MB pro zus"atzlichem RPi-Knoten erfolgt. Hierbei handelt es sich nat"urlich nicht um den Durchsatz des Netzwerks, sondern um den Durchsatz von Hauptspeicherzugriffen jedes einzelnen RPi-Knotens. Diese Steigerung tritt auch bei $n>17$
RPi-Knoten v"ollig regul"ar ein, d.h. der Durchsatz von Hauptspeicherzugriffen verh"alt sich erwartungsgem"a\ss\ f"ur alle \textit{n}. Da zudem durch das Kopieren der STREAM-Binaries in eine lokale Partition kein verbessertes Skalierungsverhalten erreicht werden konnte, scheidet ein "Uberschreiten des maximalen Netzwerk-Datendurchsatzes als Ursache aus. 
\textbf{\item Ausf"uhrung des Benchmarks.}\\
Hier sind zwei Ursachen denkbar: Netz-Dateisystem und MPI-Implementierung.

Die parallele Ausf"uhrung von STREAM wird durch MPICH angesto\ss en, das auf den im Machinefile spezifizierten CPUs die dort bestimmte Anzahl an Programmaufrufen ausf"uhrt (vgl. Kap. \ref{Versuchsaufbau} und \ref{Bramble-Versuchsaufbau}). Binaries und verwendete Bibliotheken der Benchmarks liegen im geteilten Verzeichnis \texttt{/srv} und es schien denkbar, dass mehr als 17 parallele Zugriffe das Netz-Dateisystem "uberlasten. Deswegen wurde probeweise auf den SD-Karten aller RPi-Knoten eine neue Partition erstellt und der Inhalt des geteilten Verzeichnisses darauf kopiert. In der Datei \texttt{/etc/fstab} aller RPi-Knoten wurde der Mountpoint tempor"ar entsprechend ver"andert. Messwerte f"ur den STREAM-Benchmark f"uhrten jedoch zu Performance-Einbr"uchen schon ab $n=13$ RPi-Knoten, sodass diese Ursache ausgeschlossen werden konnte.  

Ein weiterer Erkl"arungsansatz war die Funktionsweise von MPICH bzw. eine m"ogliche "Uberlastung des Netzwerks durch den MPI-Overhead ab einem Schwellenwert von $n=18$ parallelen Programmaufrufen. 

MPI hat verschiedene M"oglichkeiten der Interprozesskommunikation, unter anderem Broadcast- und Unicast-Nachrichten. Bei der parallelen Ausf"uhrung von Anwendungen kommen h"aufig die Funktionen \texttt{MPI\_Bcast} und \texttt{MPI\_Gather} zur Anwendung, bei denen ein Root-Prozess entweder eine Nachricht an alle beteiligten CPUs bzw. Prozesse schickt oder die Daten aller beteiligten CPUs bzw. Prozesse einsammelt (vgl. \ref{pie12}). Es erschien denkbar, dass eine Broadcast-Nachricht an mehr als 17 Prozessoren das Netzwerk "uberlastet, sodass es zu den beobachteten Performance-Einbr"uchen kommt.  
% TODO: streamtest.f kompilieren + Resultat
\end{enumerate}

\section{Stromverbrauch}

\section{Grenzen des Versuchsaufbaus}\label{Grenzen}

W"ahrend der Versuchsdurchf"uhrung zeigte sich, dass die jetzige Architektur des Bramble in einigen Punkten an ihre Grenzen st"o\ss t. Bevor die Benchmarks bzw. die ExperimentSuites ausgef"uhrt werden konnten, mussten einige Fehlerf"alle untersucht und behoben werden: 
\begin{enumerate}\bfseries
	\item Defekte Hardware.\\
\normalfont
Einige Mini-USB-Kabel waren bereits zu Beginn defekt. Sie mussten durch funktionsf"ahige Kabel ersetzt werden. Ebenso wurden w"ahrend der Versuchsdurchf"uhrung zwei SD-Karten unbrauchbar. Um die entsprechenden RPi-Knoten wieder lauff"ahig zu machen, musste ein Speicherabbild einer anderen SD-Karte erstellt und auf zwei neue SD-Karten geschrieben werden. In der Datei \texttt{/etc/hostname} musste der richtige DNS-Name der RPi-Knoten wieder eingetragen werden. 
	\textbf{\item RPi-Knoten nicht erreichbar (\texttt{ping}).}\\
H"aufig reagierte ein einzelner RPi-Knoten nicht auf ein \texttt{ping} von einem anderen RPi-Knoten oder von \texttt{careme} aus (Fehlermeldung: \texttt{Destination Host Unreachable}), obwohl die Status-LED Netzwerkaktivit"at anzeigte. Als einzige L"osung erwies sich Ziehen und wieder Einstecken des Mini-USB-Kabels; war das nicht erfolgreich, musste der Vorgang mit dem Netzwerkkabel wiederholt werden (das Netzwerkkabel alleine reicht nicht aus). Nach einigen Minuten war der Zielknoten i.d.R. wieder mit \texttt{ping} erreichbar. 
	\textbf{\item RPi-Knoten nicht erreichbar (\texttt{ssh}).}\\ 
Hier traten drei Fehlerf"alle auf: Am h"aufigsten war die Fehlermeldung \texttt{No route to host} beim Versuch, von \texttt{careme} oder einem anderen RPi-Knoten aus eine SSH-Verbindung aufzubauen. Auch hier musste das Netzwerkkabel gezogen und wieder eingesteckt werden. Nach einigen Minuten war der Zielknoten wieder erreichbar. 

Ein weiterer Fehlerfall war ein "uberraschender Passwortprompt f"ur \texttt{root} beim Versuch, eine SSH-Verbindung von einem RPi-Knoten aus zu einem anderen aufzubauen. Dieses Problem lie\ss\ sich durch Eintragen des RSA Public Key von \texttt{root} in die Datei \texttt{\textasciitilde/.ssh/authorized\_keys} auf dem Zielknoten l"osen. 

Seltener trat die Fehlermeldung \texttt{WARNING: REMOTE HOST IDENTIFICATION HAS CHAN\-GED!} auf. In diesem Fall musste der RSA Public Key des anfragenden RPi-Knotens in der Datei \texttt{\textasciitilde/.ssh/known\_hosts} auf dem Zielknoten korrigiert werden. 
	\textbf{\item Geteiltes Verzeichnis nicht eingeh"angt.}\\ 
Beim Neustart eines RPi-Knotens wurde h"aufig das geteilte Verzeichnis nicht eingeh"angt (Fehlermeldung z.B. \texttt{-bash: /srv/libraries/etc/.sharedprofile: No such file or directory}). Dann musste das Verzeichnis mit \texttt{mount /srv} auf dem betreffenden RPi-Knoten eingeh"angt werden. 
	\textbf{\item Bash-Befehle werden nicht erkannt.} 
Gelegentlich wurden auf einzelnen RPi-Knoten h"aufig verwendete Bash-Befehle nicht mehr erkannt (Fehlermeldung z.B. \texttt{mpiexec: command not found}). Als L"osung erwies sich ein Logout und erneuter Login auf dem betreffenden RPi-Knoten. 
\end{enumerate}
\noindent 
Der Ausschluss dieser Fehlerf"alle wurde weitestgehend in die Vorbereitung einer ExperimentSuite integriert: "Uberpr"ufung der Netzverbindung aller RPi-Knoten mit \texttt{ping} vom Berechnungsknoten \texttt{pi03} aus, Einh"angen des geteilten Verzeichnisses und damit gleichzeitig die  "Uberpr"ufung der SSH-Verbindung zu allen RPi-Knoten (vgl. Kap. \ref{Bramble-Versuchsaufbau}). 

Spontan auftretende Fehler konnten nicht im Vorfeld ausgeschlossen werden. Die Durchf"uhrung einer ExperimentSuite erforderte daher immer die Anwesenheit einer Aufsichtsperson, um z.B. Kabel und Status-LEDs der RPi-Knoten zu "uberpr"ufen. Trat ein spontaner Fehler auf, musste die Durchf"uhrung der ExperimentSuite abgebrochen und von Neuem mit den Vorbereitungen und der Durchf"uhrung begonnen werden. Auch das Trennen nicht mehr aktiver RPi-Knoten vom Stromnetz musste manuell erfolgen. Jeder Neustart eines RPi-Knoten musste eebenfalls von Hand veranlasst werden, da nur durch Ziehen und erneutes Einstecken des Mini-USB-Kabels ein Neustart veranlasst werden kann. 

Die physische Architektur des Bramble erwies sich in Teilen als Hindernis. Der Aufbau ist so beengt, dass die st"andig ben"otigten Ethernet- und Mini-USB-Anschl"usse der RPi-Knoten \texttt{pi11} -- \texttt{pi20} nur schwer zug"anglich sind. Zudem sind manche Mini-USB-Kabel so kurz abgemessen, dass sie kaum bis zum entsprechenden Knoten reichen. Ein h"aufiges Trennen und wieder Einstecken von Mini-USB- und Netzwerkkabel, wie z.B. in den ExperimentSuites zum Messen des Stromverbrauchs einmal pro RPi-Knoten und Benchmark vorgesehen, wird dadurch erschwert. 

Die vorliegenden Messergebnisse wurden durch diese Gegebenheiten nicht beeinflusst, da bei St"orungen die ExperimentSuite abgebrochen und der Versuch von Neuem begonnen wurde. F"ur zuk"unftige Versuchsaufbauten erscheint eine entsprechende Anpassung der Cluster-Architektur sinnvoll (vgl. Kap. \ref{Kap5}).
\endinput 