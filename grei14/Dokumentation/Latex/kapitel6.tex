\chapter{Anhang}\label{Anhang}

\section{Shellskripte}\label{skripte}

\subsubsection{startBenchmarks.sh}

\begin{verbatim}
#!/bin/bash

# Start script for execution of n benchmark scripts on 20 RPi-Nodes

# remove old machinefile
rm -f /srv/libraries/etc/mpich-3.0.4-shared/machinefile

touch /srv/libraries/etc/mpich-3.0.4-shared/machinefile
for host in {pi0{1..9},pi{10..20}}
do 
	echo "$host" >> /srv/libraries/etc/mpich-3.0.4-shared/machinefile
done 

# mount shared directory /srv on all RPis
for host in {pi0{1..9},pi{10..20}}
do 
	ssh root@$host 'mount /srv'
done

# navigate to execution directory
cd /srv/experimentsuite/ 

# loop over n benchmarks 
for benchmark in STREAM hpl-2.1
do 
	./$benchmark.sh  
done 
\end{verbatim}

\subsubsection{loadGeneratorConfigHpl.sh}

\begin{verbatim}
#!/bin/bash 

# Setup load generator configuration for hplinpack 
# For documentation purpose 
# For new setup: change 'hpl-config' into some other string

ssh rpi-user@careme<<'ENDSSH'                                                                                                                                               
# mysql -B: don't use history file, disable interactive behavior                                                                                                           
# mysql -s: silent mode                                                                                                                                                                
# mysql -e: execute query and exit

# define load generator                                                                                                                                                    
mysql --user=rpi-user --password=rpiWerte rpiWerte -Bse "INSERT INTO 
LoadGenerator (name,description) VALUES ('hpl-2.1','Benchmark')"                                    

# define load generator configuration key                                                                                                                                  
mysql --user=rpi-user --password=rpiWerte rpiWerte -Bse "INSERT INTO 
ENUM_LoadGeneratorConfigurationKey (\`value\`) VALUES ('hpl-config')"                               

# assign load generator configuration key to load generator configuration                                                                                                  
mysql --user=rpi-user --password=rpiWerte rpiWerte -Bse "INSERT INTO 
LoadGeneratorConfiguration (\`key\`) VALUES ('hpl-config')"                                            
ENDSSH
\end{verbatim}

\subsubsection{loadGeneratorConfigStream.sh}

\begin{verbatim}
#!/bin/bash 

# Setup load generator configuration for STREAM  
# For documentation purpose 
# For new setup: change 'stream-config' into some other string

ssh rpi-user@careme<<'ENDSSH'                                                                                                                                               
# mysql -B: don't use history file, disable interactive behavior                                                                                                           
# mysql -s: silent mode                                                                                                                                                                
# mysql -e: execute query and exit

# define load generator                                                                                                                                                    
mysql --user=rpi-user --password=rpiWerte rpiWerte -Bse "INSERT INTO 
LoadGenerator (name,description) VALUES ('STREAM','Benchmark')"                                    

# define load generator configuration key                                                                                                                                  
mysql --user=rpi-user --password=rpiWerte rpiWerte -Bse "INSERT INTO 
ENUM_LoadGeneratorConfigurationKey (\`value\`) VALUES ('stream-config')"                               

# assign load generator configuration key to load generator configuration                                                                                                  
mysql --user=rpi-user --password=rpiWerte rpiWerte -Bse "INSERT INTO 
LoadGeneratorConfiguration (\`key\`) VALUES ('stream-config')"                                            
ENDSSH
\end{verbatim}

\subsubsection{hpl-2.1.sh}

\begin{verbatim}
#!/bin/bash 

# Run Hplinpack on n downto 4 RPi-nodes, run again and shutdown RPi-node i afterwards 

# create output files                                                                 
touch results/hpl-2.1_`date +%y%m%d`.txt results/hpl-2.1_shutdown`date +%y%m%d`.txt

# loop over n RPis downto 4                                                           
for host in pi20 pi19 pi18 pi17 pi16 pi15 pi14 pi13 pi12 pi11 pi10 pi09 pi08
do
    n=${host/pi/}                                         
    n=$(echo $n|sed 's/^0*//')
    echo "Number of active RPis: $n"
    echo "Number of powered RPis: 20"
    starttime=`date +%s`
    echo "Start time: $starttime"

# Setup experiment suite in database
# One experiment suite for every program run needed 

    ssh rpi-user@careme<<ENDSSH

# initialize experiment suite 
	mysql -u rpi-user -prpiWerte rpiWerte -Bse "INSERT INTO 
ExperimentSuite (granularityLevel,objective,executionStartedAt) VALUES 
('Cluster Blackbox','experiment no1',$starttime)"
ENDSSH

# get experiment suite id 
ssh rpi-user@careme<<ENDSSH                                                                                                                                                 
	touch /tmp/myid.txt                                                                                                                                                         
	mysql -u rpi-user -prpiWerte rpiWerte -Bse "SELECT id FROM 
ExperimentSuite WHERE executionStartedAt=$starttime" > /tmp/myid.txt                                             
ENDSSH

# read in experiment suite id from tmp file on careme 
myid=$(ssh rpi-user@careme "cat /tmp/myid.txt")

# remove tmp file from careme                                                                                                                                               
ssh rpi-user@careme "rm /tmp/myid.txt"

ssh rpi-user@careme<<ENDSSH                                                                                                                                                 
# insert number of active RPis                                                                                                                                              
	mysql --user=rpi-user --password=rpiWerte rpiWerte -Bse 
"INSERT INTO ExperimentSuiteConfiguration (\\\`key\\\`,\\\`value\\\`,
experimentSuiteId) VALUES ('NumberOfActiveRPis','${n}',$myid)"                                                                                                                                                            
                                                                                                                                                                            
# insert number of powered RPis                                                                                                                                             
mysql --user=rpi-user --password=rpiWerte rpiWerte -Bse 
"INSERT INTO ExperimentSuiteConfiguration (\\\`key\\\`,\\\`value\\\`,
experimentSuiteId) VALUES ('NumberOfPoweredRPis','14',$myid)"                                                                                                                                                             
                                                                                                                                                                            
# map load generator configuration to experiment suite                                                                                                                      
	mysql --user=rpi-user --password=rpiWerte rpiWerte -Bse 
"INSERT INTO N2M_loadGConf2expSuite (loadGeneratorConfigurationId,experimentSuiteId) 
VALUES (6,$myid)"              
ENDSSH

echo "Database setup complete"

# log number of active RPis/powered RPis to results/hpl-2.1_`date +%y%m%d`.txt
echo "Active RPis: "$n"/Powered RPis: 20" >> results/hpl-2.1_`date +%y%m%d`.txt

# start benchmark on n RPis, log to results/hpl-2.1_`date +%y%m%d`.txt                        
    mpiexec -n $n -machinefile /srv/libraries/etc/mpich-3.0.4-shared/machinefile 
-wdir /srv/benchmarks/bin/hpl-2.1 /srv/benchmarks/bin/hpl-2.1/xhpl >> 
results/hpl-2.1_`date +%y%m%d`.txt

    echo "HPlinpack finished on $n RPis active, 20 RPis powered" 
done 

# loop over n RPis downto 4                                                           
for host in pi20 pi19 pi18 pi17 pi16 pi15 pi14 pi13 pi12 pi11 pi10 pi09 pi08
do
    n=${host/pi/}                                         
    n=$(echo $n|sed 's/^0*//')
    echo "Number of active RPis: $n"
    echo "Number of powered RPis: $n"
    starttime=`date +%s`
    echo "Start time: $starttime"

# Setup experiment suite in database
# One experiment suite for every program run needed 

# initialize experiment suite
    ssh rpi-user@careme<<ENDSSH                                                                                                                                                 

# initialize experiment suite                                                                                                                                               
	mysql -u rpi-user -prpiWerte rpiWerte -Bse "INSERT INTO 
ExperimentSuite (granularityLevel,objective,executionStartedAt) 
VALUES ('Cluster Blackbox','experiment no2',$starttime)"                                                                                                                                                                          
ENDSSH
                                                                                                                                                       

# get experiment suite id 
ssh rpi-user@careme<<ENDSSH                                                                                                                                                 
	touch /tmp/myid.txt                                                                                                                                                         
	mysql -u rpi-user -prpiWerte rpiWerte -Bse "SELECT id FROM 
ExperimentSuite WHERE executionStartedAt=$starttime" > /tmp/myid.txt                                             
ENDSSH

# read in experiment suite id from tmp file on careme 
myid=$(ssh rpi-user@careme "cat /tmp/myid.txt")
# echo $myid

# remove tmp file from careme                                                                                                                                               
ssh rpi-user@careme "rm /tmp/myid.txt"

ssh rpi-user@careme<<ENDSSH                                                                                                                                                 
# insert number of active RPis                                                                                                                                              
	mysql --user=rpi-user --password=rpiWerte rpiWerte -Bse "INSERT INTO 
ExperimentSuiteConfiguration (\\\`key\\\`,\\\`value\\\`,experimentSuiteId) 
VALUES ('NumberOfActiveRPis','${n}',$myid)"                                                                                                                                                            
                                                                                                                                                                            
# insert number of powered RPis                                                                                                                                             
	mysql --user=rpi-user --password=rpiWerte rpiWerte -Bse "INSERT INTO 
ExperimentSuiteConfiguration (\\\`key\\\`,\\\`value\\\`,experimentSuiteId) 
VALUES ('NumberOfPoweredRPis','14',$myid)"                                                                                                                                                             
                                                                                                                                                                            
# map load generator configuration to experiment suite                                                                                                                      
	mysql --user=rpi-user --password=rpiWerte rpiWerte -Bse "INSERT INTO 
N2M_loadGConf2expSuite (loadGeneratorConfigurationId,experimentSuiteId) 
VALUES (6,$myid)"              
ENDSSH

echo "Database setup complete"

# log number of active RPis/powered RPis to results/hpl-2.1_`date +%y%m%d`.txt
echo "Active RPis: "$n"/Powered RPis: 20" >> results/hpl-2.1_shutdown`date +%y%m%d`.txt

# start benchmark on n RPis, log to results/hpl-2.1_shutdown`date +%y%m%d`.txt                        
    mpiexec -n $n -machinefile /srv/libraries/etc/mpich-3.0.4-shared/machinefile 
-wdir /srv/benchmarks/bin/hpl-2.1 /srv/benchmarks/bin/hpl-2.1/xhpl 
>> results/hpl-2.1_shutdown`date +%y%m%d`.txt
    echo "HPlinpack finished on $n RPis active, $n RPis powered"
    ssh $host 'shutdown -hP 0'
done 

# create input file for database                                                                                                                                            
touch results/hpl-2.1_db`date +%y%m%d`.txt

# find all lines in first output file beginning with 'WR' (result lines) 
# and print together with 3 following lines (timestamp lines)                                       
grep '^WR' results/hpl-2.1_`date +%y%m%d`.txt -A 3 
> results/hpl-2.1_db`date +%y%m%d`.txt

# find all lines in second output file beginning with 'WR' (result lines) 
# and print together with 3 following lines (timestamp lines)                                      
grep '^WR' results/hpl-2.1_shutdown`date +%y%m%d`.txt -A 3 
>> results/hpl-2.1_db`date +%y%m%d`.txt

# remove all empty lines                                                                                                                                                    
sed -i '/^$/d' results/hpl-2.1_db`date +%y%m%d`.txt

# remove all lines containing "start" (only "end" needed)                                                                                                                   
sed -i '/start/d' results/hpl-2.1_db`date +%y%m%d`.txt

# remove all lines containing only '--' (result from grep -A)                                                                                                               
sed -i '/--/d' results/hpl-2.1_db`date +%y%m%d`.txt

# transform database input file to lines containing space seperated values                                                                                                  
sed -i 's/ \{2,\}/ /g' results/hpl-2.1_db`date +%y%m%d`.txt
sed -i 'N;s/\n/ /' results/hpl-2.1_db`date +%y%m%d`.txt
                                                                                                                                              
ssh rpi-user@careme<<'ENDSSH'                                                                                                                 
cat /srv/nfs-share/experimentsuite/results/hpl-2.1_db`date +%y%m%d`.txt | 
cut -d' ' -f6,7,12,13,14,15 | while read line                                                     
do                                                                                                                                                                          
    arr=($line)                                                                                                                                                             
    time=${arr[0]}                                                                                                                                                          
    echo "Time: $time"                                                                                                                                                      
    gflops=${arr[1]}                                                                                                                                                        
    echo "Gflops: $gflops"                                                                                                                                                  
    timestamp=${arr[@]:2}                                                                                                                                                   
    unixtime=$(date -d "${timestamp}" "+%s")                                                                                                                                
    echo $unixtime                                                                                                                                                          
    mysql --user=rpi-user --password=rpiWerte rpiWerte -Bse 
    "INSERT INTO MeasurementValue (parameter,\`value\`,measuredAt) 
    VALUES ('Time',$time,$unixtime)"                
    mysql --user=rpi-user --password=rpiWerte rpiWerte -Bse 
    "INSERT INTO MeasurementValue (parameter,\`value\`,measuredAt) 
    VALUES ('Gflops',$gflops,$unixtime)"
done
ENDSSH
\end{verbatim}

\subsubsection{STREAM.sh}

\begin{verbatim}
#!/bin/bash                                                                           

# Run STREAM on1 RPi-Nodes, run again and shutdown RPi-Node n a

# create output files                                                                 
touch results/STREAM_`date +%y%m%d`.txt results/STREAM_shutdown`date +%y%m%d`.txt

# loop over n RPis 
for host in pi20 pi19 pi18 pi17 pi16 pi15 pi14 pi13 
pi12 pi11 pi10 pi09 pi08 pi07 pi06 pi05 
do
    n=${host/pi/}
    n=$(echo $n|sed 's/^0*//')
    n=$(($n - 4))
#    if [ $n > 0 ]; then 
	echo "Number of active RPis: $n"
	echo "Number of powered RPis: 16"
	starttime=`date +%s`
	echo "Start time: $starttime"

# Setup experiment suite in database                                                                                                                                        
# One experiment suite for every program run needed

	ssh rpi-user@careme<<ENDSSH
# initialize experiment suite                                                                                                                                               
	mysql -u rpi-user -prpiWerte rpiWerte -Bse 
"INSERT INTO ExperimentSuite (granularityLevel,objective,executionStartedAt) 
VALUES ('Cluster Blackbox','experiment no3',$starttime)"                                                                                                                                                              
ENDSSH

# get experiment suite id                                                                                                                                                   
	ssh rpi-user@careme<<ENDSSH                                                                                                                                                                                                                                                                                                                            
	touch /tmp/myid.txt                                                                                                                                                                                                                                                                                                                                 
	mysql -u rpi-user -prpiWerte rpiWerte -Bse 
"SELECT id FROM ExperimentSuite WHERE executionStartedAt=$starttime" > /tmp/myid.txt                                            
ENDSSH

# read in experiment suite id from tmp file on careme                                                                                                                       
	myid=$(ssh rpi-user@careme "cat /tmp/myid.txt")
# echo $myid                                                                                                                                                                

# remove tmp file from careme                                                                                                                                                                                                                                                                                                                          
	ssh rpi-user@careme "rm /tmp/myid.txt"
	ssh rpi-user@careme<<ENDSSH                                                                                                                                                
                                                                                                                                                                            
# insert number of active RPis
	mysql --user=rpi-user --password=rpiWerte rpiWerte -Bse 
"INSERT INTO ExperimentSuiteConfiguration (\\\`key\\\`,\\\`value\\\`,
experimentSuiteId) VALUES ('NumberOfActiveRPis','${n}',$myid)"                                                                                                                                                           

# insert number of powered RPis                                 
	mysql --user=rpi-user --password=rpiWerte rpiWerte -Bse 
"INSERT INTO ExperimentSuiteConfiguration (\\\`key\\\`,\\\`value\\\`,
experimentSuiteId) VALUES ('NumberOfPoweredRPis','16',$myid)"                                                                                                                                                            

# map load generator configuration to experiment suite                                                                                                                     
	mysql --user=rpi-user --password=rpiWerte rpiWerte -Bse 
"INSERT INTO N2M_loadGConf2expSuite (loadGeneratorConfigurationId,experimentSuiteId) 
VALUES (7,$myid)"             
ENDSSH

	echo "Database setup complete"

# start benchmark on n RPis, log to results/STREAM_`date +%y%m%d`.txt                                                                                                      
	mpiexec -n $n -machinefile /srv/libraries/etc/mpich-3.0.4-shared/machinefile 
-wdir /srv/benchmarks/bin/STREAM /srv/benchmarks/bin/STREAM/stream 
>> results/STREAM_`date +%y%m%d`.txt

# add unix timestamp to results file
	echo "Unixtime: `date +%s`" >> results/STREAM_`date +%y%m%d`.txt

	echo "STREAM finished on $n RPis active, 16 RPis powered"
done


# loop over n RPis 
for host in pi20 pi19 pi18 pi17 pi16 pi15 pi14 pi13 
pi12 pi11 pi10 pi09 pi08 pi07 pi06 pi05 
do
    n=${host/pi/}
    n=$(echo $n|sed 's/^0*//')
    n=$(($n - 4))
#    if [ $n > 0 ]; then 
	echo $n
	echo "Number of active RPis: $n"
	echo "Number of powered RPis: $n"
	starttime=`date +%s`
	echo "Start time: $starttime"

# Setup experiment suite in database                                                                                                                                        
# One experiment suite for every program run needed
	ssh rpi-user@careme<<ENDSSH
# initialize experiment suite                                                                                                                                               
	mysql -u rpi-user -prpiWerte rpiWerte -Bse 
"INSERT INTO ExperimentSuite (granularityLevel,objective,executionStartedAt) 
VALUES ('Cluster Blackbox','experiment no4',$starttime)"                                                                                                                                                              
ENDSSH

# get experiment suite id                                                                                                                                                   
	ssh rpi-user@careme<<ENDSSH                                                                                                                                                                                                                                                                                                                            
t	ouch /tmp/myid.txt                                                                                                                                                                                                                                                                                                                                 
	mysql -u rpi-user -prpiWerte rpiWerte -Bse 
"SELECT id FROM ExperimentSuite WHERE executionStartedAt=$starttime" > /tmp/myid.txt                                            
ENDSSH

# read in experiment suite id from tmp file on careme                                                                                                                       
	myid=$(ssh rpi-user@careme "cat /tmp/myid.txt")

# remove tmp file from careme                                                                                                                                                                                                                                                                                                                          
	ssh rpi-user@careme "rm /tmp/myid.txt"
    
	ssh rpi-user@careme<<ENDSSH                                                                                                                                                
        # insert number of active RPis
    mysql --user=rpi-user --password=rpiWerte rpiWerte -Bse 
"INSERT INTO ExperimentSuiteConfiguration (\\\`key\\\`,\\\`value\\\`,experimentSuiteId) 
VALUES ('NumberOfActiveRPis','${n}',$myid)"                                                                                                                                                           

    # insert number of powered RPis                                 
    mysql --user=rpi-user --password=rpiWerte rpiWerte -Bse 
"INSERT INTO ExperimentSuiteConfiguration (\\\`key\\\`,\\\`value\\\`,experimentSuiteId) 
VALUES ('NumberOfPoweredRPis','${n}',$myid)"                                                                                                                                                            

    # map load generator configuration to experiment suite                                                                                                                     
    mysql --user=rpi-user --password=rpiWerte rpiWerte -Bse 
"INSERT INTO N2M_loadGConf2expSuite (loadGeneratorConfigurationId,experimentSuiteId) 
VALUES (7,$myid)"             
ENDSSH

	echo "Database setup complete"
# log number of active RPis/powered RPis to results/STREAM_`date +%y%m%d`.txt 

# start benchmark on n RPis, log to results/STREAM_`date +%y%m%d`.txt                                                                                                      
	mpiexec -n $n -machinefile /srv/libraries/etc/mpich-3.0.4-shared/machinefile 
-wdir /srv/benchmarks/bin/STREAM /srv/benchmarks/bin/STREAM/stream 
>> results/STREAM_shutdown`date +%y%m%d`.txt

	ssh $host 'shutdown -hP 0'
    
# add unix timestamp to results file
	echo "Unixtime: `date +%s`" >> results/STREAM_shutdown`date +%y%m%d`.txt
	echo "STREAM finished on $n RPis active, $n RPis powered"

done

# create input file for database                                                                                                                                            
touch results/STREAM_db`date +%y%m%d`.txt

# find all lines in first output file beginning with 'Copy' (first result line) 
# and print together with 7 following lines
grep '^Copy' results/STREAM_`date +%y%m%d`.txt -A 7 
> results/STREAM_db`date +%y%m%d`.txt

# find all lines in second output file beginning with 'Copy' (result lines) 
# and print together with 7 following lines (timestamp lines)                                    
grep '^Copy' results/STREAM_shutdown`date +%y%m%d`.txt -A 7 
>> results/STREAM_db`date +%y%m%d`.txt

# remove all lines containing only '--' (result from grep -A)                                                                                                              
sed -i '/--/d' results/STREAM_db`date +%y%m%d`.txt

# remove all lines beginning with blank                                                                                                                                    
sed -i '/^ /d' results/STREAM_db`date +%y%m%d`.txt

# insert new line delimiter before 'Copy'                                                                                                                                  
sed -i 's/Copy/|Copy/' results/STREAM_db`date +%y%m%d`.txt

# transform to lines containing space seperated values                                                                                                                     
sed -i 's/ \{2,\}/ /g' results/STREAM_db`date +%y%m%d`.txt

# transform whole file into one line (for read line)                                                                                                                       
sed -i ':a;N;$!ba;s/\n/ /g' results/STREAM_db`date +%y%m%d`.txt

# substitute '|' with '\n' (new line delimiter)                                                                                                                            
sed -i 's/|/\n/g' results/STREAM_db`date +%y%m%d`.txt

# remove empty first line                                                                                                                                                  
sed -i '/^$/d' results/STREAM_db`date +%y%m%d`.txt

ssh rpi-user@careme<<'ENDSSH'                                                                                                                                               
cat /srv/nfs-share/experimentsuite/results/STREAM_db`date +%y%m%d`.txt | 
cut -d' ' -f2,3,7,8,12,13,17,18,22 | while read line                                                    
do                                                                                                                                                                          
    # echo $line                                                                                                                                                            
    arr=($line)                                                                                                                                                            
    copy_rate=${arr[0]}                                                                                                                                                     
    # echo $copy_rate                                                                                                                                                       
    copy_time=${arr[1]}                                                                                                                                                     
    # echo $copy_time                                                                                                                                                       
    scale_rate=${arr[2]}                                                                                                                                                    
    # echo $scale_rate                                                                                                                                                      
    scale_time=${arr[3]}                                                                                                                                                    
    # echo $scale_time                                                                                                                                                      
    add_rate=${arr[4]}                                                                                                                                                      
    # echo $add_rate                                                                                                                                                        
    add_time=${arr[5]}
    # echo $add_time                                                                                                                                                        
    triad_rate=${arr[6]}                                                                                                                                                    
    # echo $triad_rate                                                                                                                                                      
    triad_time=${arr[7]}                                                                                                                                                    
    # echo $triad_time                                                                                                                                                      
    unixtime=${arr[8]}                                                                                                                                                      
    # echo $unixtime                                                                                                                                                        
    mysql --user=rpi-user --password=rpiWerte rpiWerte -Bse 
"INSERT INTO MeasurementValue (parameter,\`value\`,measuredAt) 
VALUES ('Copy/Rate',$copy_rate,$unixtime)"       
    mysql --user=rpi-user --password=rpiWerte rpiWerte -Bse 
"INSERT INTO MeasurementValue (parameter,\`value\`,measuredAt) 
VALUES ('Copy/Avg time',$copy_time,$unixtime)"   
    mysql --user=rpi-user --password=rpiWerte rpiWerte -Bse 
"INSERT INTO MeasurementValue (parameter,\`value\`,measuredAt) 
VALUES ('Scale/Rate',$scale_rate,$unixtime)"     
    mysql --user=rpi-user --password=rpiWerte rpiWerte -Bse 
"INSERT INTO MeasurementValue (parameter,\`value\`,measuredAt) 
VALUES ('Scale/Avg time',$scale_time,$unixtime)"
    mysql --user=rpi-user --password=rpiWerte rpiWerte -Bse 
"INSERT INTO MeasurementValue (parameter,\`value\`,measuredAt) 
VALUES ('Add/Rate',$add_rate,$unixtime)"         
    mysql --user=rpi-user --password=rpiWerte rpiWerte -Bse 
"INSERT INTO MeasurementValue (parameter,\`value\`,measuredAt) 
VALUES ('Add/Avg time',$add_time,$unixtime)"     
    mysql --user=rpi-user --password=rpiWerte rpiWerte -Bse 
"INSERT INTO MeasurementValue (parameter,\`value\`,measuredAt) 
VALUES ('Triad/Rate',$triad_rate,$unixtime)"     
    mysql --user=rpi-user --password=rpiWerte rpiWerte -Bse 
"INSERT INTO MeasurementValue (parameter,\`value\`,measuredAt) 
VALUES ('Triad/Avg time',$triad_time,$unixtime)"
done                                                                                                                                                                        
ENDSSH
\end{verbatim}

\section{Ergebnisse der ausgew"ahlten Benchmarks auf dem RPi-Einzelrechner}\label{rpi-anhang}

\subsection{Linpack 100}\label{rpi-linpack}

\begin{verbatim}
##########################################
Unrolled Double Precision Linpack Benchmark - Linux Version in 'C/C++'

Optimisation Opt 3 32 Bit

norm resid      resid           machep         x[0]-1          x[n-1]-1
   1.7    7.41628980e-14   2.22044605e-16  -1.49880108e-14  -1.89848137e-14

Times are reported for matrices of order          100
1 pass times for array with leading dimension of  201

      dgefa      dgesl      total     Mflops       unit      ratio
    0.01613    0.00057    0.01669      41.14     0.0486     0.2981

Calculating matgen overhead
        10 times   0.01 seconds
       100 times   0.14 seconds
       200 times   0.28 seconds
       400 times   0.57 seconds
       800 times   1.13 seconds
Overhead for 1 matgen      0.00141 seconds

Calculating matgen/dgefa passes for 1 seconds
        10 times   0.17 seconds
        20 times   0.35 seconds
        40 times   0.70 seconds
        80 times   1.40 seconds
Passes used         57 

Times for array with leading dimension of 201

      dgefa      dgesl      total     Mflops       unit      ratio
    0.01609    0.00054    0.01663      41.29     0.0484     0.2970
    0.01603    0.00054    0.01658      41.43     0.0483     0.2960
    0.01610    0.00054    0.01664      41.25     0.0485     0.2972
    0.01609    0.00054    0.01663      41.29     0.0484     0.2970
    0.01603    0.00061    0.01663      41.28     0.0484     0.2970
Average                                41.31

Calculating matgen2 overhead
Overhead for 1 matgen      0.00137 seconds

Times for array with leading dimension of 200

      dgefa      dgesl      total     Mflops       unit      ratio
    0.01447    0.00054    0.01502      45.73     0.0437     0.2682
    0.01437    0.00051    0.01489      46.13     0.0434     0.2658
    0.01447    0.00051    0.01498      45.84     0.0436     0.2675
    0.01445    0.00051    0.01496      45.89     0.0436     0.2672
    0.01441    0.00051    0.01492      46.02     0.0435     0.2665
Average                                45.92

##########################################

From File /proc/cpuinfo
Processor : ARMv6-compatible processor rev 7 (v6l)
BogoMIPS  : 697.95
Features  : swp half thumb fastmult vfp edsp java tls 
CPU implementer	: 0x41
CPU architecture: 7
CPU variant	  : 0x0
CPU part	  : 0xb76
CPU revision	  : 7

Hardware	  : BCM2708
Revision	  : 000f
Serial		    : 00000000e98379f1


From File /proc/version
Linux version 3.6.11+ (dc4@dc4-arm-01) (gcc version 4.7.2 20120731 (prerelease) 
(crosstool-NG linaro-1.13.1+bzr2458 - Linaro GCC 2012.08) ) 
#538 PREEMPT Fri Aug 30 20:42:08 BST 2013


Unrolled Double  Precision       41.31 Mflops 

##############################################
\end{verbatim}

\subsection{Whetstone}\label{rpi-whetstone}

\begin{verbatim}
##########################################
Single Precision C Whetstone Benchmark Opt 3 32 Bit, Sat Nov 30 15:22:14 2013

Calibrate
       0.04 Seconds          1   Passes (x 100)
       0.19 Seconds          5   Passes (x 100)
       0.96 Seconds         25   Passes (x 100)
       4.80 Seconds        125   Passes (x 100)

Use 260  passes (x 100)

From File /proc/cpuinfo
Processor : ARMv6-compatible processor rev 7 (v6l)
BogoMIPS  : 697.95
Features  : swp half thumb fastmult vfp edsp java tls 
CPU implementer	: 0x41
CPU architecture: 7
CPU variant	  : 0x0
CPU part	  : 0xb76
CPU revision	  : 7

Hardware	  : BCM2708
Revision	  : 000f
Serial		    : 00000000e98379f1


From File /proc/version
Linux version 3.6.11+ (dc4@dc4-arm-01) (gcc version 4.7.2 20120731 (prerelease) 
(crosstool-NG linaro-1.13.1+bzr2458 - Linaro GCC 2012.08) ) 
#538 PREEMPT Fri Aug 30 20:42:08 BST 2013


          Single Precision C/C++ Whetstone Benchmark

Loop content                  Result              MFLOPS      MOPS   Seconds

N1 floating point     -1.12475013732910156        97.643              0.051
N2 floating point     -1.12274742126464844       100.883              0.346
N3 if then else        1.00000000000000000                 690.831    0.039
N4 fixed point        12.00000000000000000                 423.573    0.193
N5 sin,cos etc.        0.49911010265350342                   5.050    4.284
N6 floating point      0.99999982118606567        86.081              1.629
N7 assignments         3.00000000000000000                 498.602    0.096
N8 exp,sqrt etc.       0.75110864639282227                   2.724    3.551

MWIPS                                            255.154             10.190

\end{verbatim}

\subsection{STREAM}\label{rpi-stream}

\begin{verbatim}
-------------------------------------------------------------
STREAM version $Revision: 5.10 $
-------------------------------------------------------------
This system uses 8 bytes per array element.
-------------------------------------------------------------
Array size = 10000000 (elements), Offset = 0 (elements)
Memory per array = 76.3 MiB (= 0.1 GiB).
Total memory required = 228.9 MiB (= 0.2 GiB).
Each kernel will be executed 10 times.
 The *best* time for each kernel (excluding the first iteration)
 will be used to compute the reported bandwidth.
-------------------------------------------------------------
Your clock granularity/precision appears to be 1 microseconds.
Each test below will take on the order of 736819 microseconds.
   (= 736819 clock ticks)
Increase the size of the arrays if this shows that
you are not getting at least 20 clock ticks per test.
-------------------------------------------------------------
WARNING -- The above is only a rough guideline.
For best results, please be sure you know the
precision of your system timer.
-------------------------------------------------------------
Function    Best Rate MB/s  Avg time     Min time     Max time
Copy:             274.4     0.586838     0.583109     0.599162
Scale:            209.3     0.766437     0.764583     0.772736
Add:              287.2     0.838107     0.835557     0.842724
Triad:            271.1     0.886793     0.885394     0.889097
-------------------------------------------------------------
Solution Validates: avg error less than 1.000000e-13 on all three arrays
-------------------------------------------------------------
\end{verbatim}
\endinput 