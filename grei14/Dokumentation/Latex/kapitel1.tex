\chapter{Einleitung}\label{Kapitel 1}

% CONTEXT OF THE PAPER

Seit Beginn seiner Entwicklung im Jahr 2009 boomt der Minicomputer Raspberry Pi (im Folgenden als \textit{RPi} bezeichnet): Er erhielt z.B. den Designpreis INDEX: award 2013 (vgl. \url{http://designtoimprovelife.dk/category/s15-award2013/index-award-winners-2013), wurde als Innovation des Jahres bei den T3 Gadget Awards 2012 ausgezeichnet (vgl. \url{http://www.t3.com/news/t3-gadget-awards-2012-award-winners)
und zum Product of the Year 2012 des Linux Journal gew"ahlt (vgl. \url{http://www.linuxjournal.com/slideshow/readers-choice-2012?page=49}).

Im Februar dieses Jahres war das Modell B "uber 2,5 Millionen Mal verkauft worden. Was sind die Gr"unde f"ur den Erfolg des RPi? 

\section{Hintergrund}\label{Hintergrund}

Die Leistung von Rechnern, seien es Gro\ss rechner, Desktop-Rechner oder Minicomputer, wird h"aufig durch Benchmarks ermittelt. Das erm"oglicht die Vergleichbarkeit der Testergebnisse unterschiedlicher Systeme. Bekannte und erprobte Benchmark-Suites sind z.B. Linpack und Whetstone, mit denen seit den 70er Jahren die Performance von Supercomputern ermittelt wird. F"ur Einzelrechner mit Linux-Systemen gibt es z.B. die Phoronix Test Suite oder UnixBench, um die Performance der einzlnen Komponenten wie CPU, GPU und RAM zu evaluieren. 

% PROBLEM STATEMENT
\section{Fragestellung}\label{Fragestellung}

Vor diesem Hintergrund stellt sich die Frage: Wie verh"alt sich ein RPi bei der Ausf"uhrung von HPC-Benchmarks? Noch interessanter ist die Untersuchung eines RPi-Clusters. Welche CPU-Performance l"asst sich damit erzielen und wie verh"alt sich der Cluster bei Hinzunahme von Ressourcen, d.h. RPi-Rechenkernen? Im Zentrum dieser Arbeit stehen daher CPU-Performance und Skalierungsverhalten eines RPi-Clusters unter den Testbedingungen ausgew"ahlter HPC-Benchmarks. Dazu wird zun"achst die die Performance eines RPi-Einzelrechners ermittelt. Anschlie\ss end wird die Performance des RPi-Clusters unter denselben Benchmarks evaluiert, wobei der Energieverbrauch im Vordergrund steht. 

% Todo: CHALLENGES
\section{Herausforderungen}\label{Herausforderungen}

Die Anpassung von HPC-Benchmarks an einen RPi und einen RPi-Cluster stellen besondere Anforderungen an den Versuchsaufbau: Es m"ussen geeignete Implementierungen der Benchmarks f"ur den RPi-Einzelrechner und den RPi-Cluster verwendet werden und die Vergleichbarkeit der Ergebnisse muss durch eine geeignete Testumgebung sicher gestellt werden.  

\section{Vorgehensweise und Struktur}\label{Struktur}
% TODO: Vorgehensweise
Um den Bezugsrahmen zu verdeutlichen, werden in Kapitel 2 zun"achst grundlegende Definitionen gekl"art. Insbesondere werden die ausgew"ahlten Benchmarks vorgestellt (vgl. Kap. \ref{Benchmarks}) und die Spezifikationen von RPi (vgl. Kap. \ref{RPi-Spezi}) und RPi-Cluster erl"autert (vgl. Kap. \ref{Bramble-Spezi}). Versuchsaufbau und -ablauf werden mit Blick auf Auswahl und Anpassung der RPi-spezifischen Parameter im folgenden Kapitel dargestellt (vgl. Kap. \ref{Kapitel 3}). Schlie\ss lich  werden die Messergebnisse auf dem RPi-Einzelrechner und dem Cluster dargestellt (vgl. Kap. \ref{Ergebnisse}) und interpretiert (vgl. Kap. \ref{Kapitel 4}). Den Abschluss bilden eine Zusammenfassung der Untersuchungsergebnisse und ein Ausblick (vgl. Kap. \ref{Kapitel 5}).

% TODO: Textbelege Grundlagenwerke, die sonst nirgends zitiert werden (ZITATE AUS LITERATURVERZEICHNIS (IM TEXT): z. B. \cite{han99})
\endinput 
